\chapter{Data Acquisition and Processing}
\label{ch:Data Acquisition and Processing}

\section{CUORE Data Acquisition}
Because of the need in CUORE to calibrate $\sim$monthly, there are multiple layers for how data is organized for CUORE.
At the largest scales, there are datasets that begin and end during these monthly calibrations where the closing calibration for one dataset can be used as the opening calibration for the following dataset.
Composing each dataset are runs of $\sim24$ hour length.
These runs are further broken down into 3 types: `background' for physics runs, `calibration' for calibration runs, and `test' runs.
The test runs can be much more varied in duration than the ``standard" 24-hour background and calibration runs, and can include runs for various reasons, such as setting the working points of the NTDs or for scanning pulse tube phases.

At the smallest scales, the events on each detector are similar to that shown in \autoref{fig:Sample_pulse}.
These pulses occur with rise and fall times $\approx100$ ms and $\approx400$, respectively \cite{Alduino:2017ehq}.
A 10-second window (3 seconds before and 7 seconds after) is taken around these events to understand a stable bolometer temperature before and after the event, and the amplitude of the event is used to define the energy of the pulse.
The event rate per detector is $\approx 50$ mHz and $\approx6$ mHz in calibration\footnote{With the performance of the internal and external calibration systems, this rate can vary significantly on various towers and crystals.} and physics data, respectively.
In addition, as noted in \autoref{ssec:Particle Detection with Bolometers}, there are additional thermal pulses generated by the Si heater on the crystals occurring every few minutes.
While there is a software trigger applied on these events, which is discussed more in \autoref{sec:Online Data Taking}, there is no hardware trigger on these events.
This is due to the simplicity of the detection method in CUORE, in that the only signals are purely thermal and no precise ($\mathcal{O}(\mu~\textrm{s}$) timing is needed.
As a result, the data can be sampled continuously for each bolometer\footnote{Of the 988 CUORE bolometers, 4 observed an issue with the connections from the thermistor to the electronics, and are unreadable.
The other bolometers have a connected signal to the thermistor, while $\sim3\%$ have non-functioning heaters.} at a rate of 1 MHz.

\section{Electronics Hardware}
\label{sec:Electronics Hardware}
On top of the Y-beam directly above the cryostat are the electronics hardware that reads out and biases the thermistor and Si heater electronic circuits.
These signals are carried up from the Cu-PEN tapes through the cryostat with twisted-pair constantin wires.
The NTDs are biased through two low-noise load resistors, and the voltage is measured by means of a low-noise room temperature preamplifier, a gain amplifier, and a 6-pole Thomson-Bessel low-pass filter \cite{doi:10.1063/1.4936269, PESSINA2000132, ARNABOLDI2010327}.
\section{Online Data Taking}
\label{sec:Online Data Taking}
We take data online?
\subsection*{Signal Triggering}
We need to trigger on pulses
\section{Offline Data Processing}
We take this data and calculate stuffs
\subsection{First-Level Processing}
\subsubsection{Preprocess}
We prepare stuff for real processing
\subsubsection{Amplitude Evaluation}
We calculate the height of the peaks
\subsubsection{Stabilization}
Need to stabilize at different baselines
\label{ssec:Stabilization}

The energy dissipated in the silicon heater transfers into the bolometer similarly how a real energy deposition would look in a physics event, and allows for us to understand the response of the detector to fixed-energy input across various detector baselines as the temperature of the detectors drifts \cite{ALESSANDRELLO1998454:Si-heater}.

\subsubsection{Calibration}
How do we tell that different calibration rates are correct?
\label{ssec:Calibration}
\subsection{Second-Level Processing}
\subsubsection{Pulse Shape Analysis}
\subsubsection{Coincidence Analysis}



