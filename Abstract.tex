\begin{abstract}
This thesis describes a search performed at the Cryogenic Underground Observatory for Rare Events (CUORE) for Majoron-emitting \zeronubb~decays.
A discovery of any form of \zeronubb~decay would be of immense important to the field of physics, as any mechanism by which \zeronubb~decay would require physics beyond the Standard Model.
In addition, \zeronubb~decays and type I Majoron models, would show that lepton number is not a conserved quantity and, along with other measurements, could explain the prevalence of matter over antimatter in the early universe.
Also described in this thesis is the experiment, CUORE, which is a neutrinoless double-beta \zeronubb~decay experiment currently in operation at the Laboratori Nazionali del Gran Sasso (LNGS) in Italy, along with a detailed description of one of the calibration systems used in the experiment.
Since April 2017, CUORE has been taking data in $^{130}$Te using 988 \teotwo~crystals arranged in 19 towers inside of a custom cryostat operating at approximately 10 mK.
The results presented in this thesis correspond to a live-time of \color{red}insert live time here\color{black} and place upper limits (best values) of \color{red}insert limits+best-fits here\color{black} (90\% CL).
These results are the strongest limits in $^{130}$Te to date, and correspond to \color{red}insert values of coupling constants?\color{black}.
Lastly, these searches for Majoron decays continue to be an interesting phase space for research into \zeronubb, and continue to be another promising decay mode to observe as detector technology and background reduction techniques improve.
\end{abstract}